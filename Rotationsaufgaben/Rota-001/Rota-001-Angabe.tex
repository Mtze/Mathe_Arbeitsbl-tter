\documentclass{scrartcl}
\usepackage{amsmath}
\usepackage{a4wide}
\usepackage{float}
\usepackage[ngerman]{babel}
\usepackage{amssymb}
\usepackage{hyperref}
\usepackage[utf8]{inputenc}
\usepackage[T1]{fontenc}
\usepackage{graphicx}
\parindent0cm

\setcounter{tocdepth}{4}
\setcounter{secnumdepth}{4}
\usepackage[headsepline]{scrpage2}

\usepackage{scrpage2}
\pagestyle{scrheadings}
\clearscrheadfoot



\usepackage{tikz} 
\usetikzlibrary{shapes.misc}

\usepackage{pstricks}
\usepackage{pst-plot}
\usepackage{pstcol}



\ihead{Matthias Linhuber}
\chead{Arbeitsblatt Mathe}
\ohead{\today}

\usepackage{geometry}
\geometry{a4paper, top=20mm, left=20mm, right=20mm, bottom=10mm,
headsep=7mm, footskip=12mm}




\begin{document}

	\section{Rotierende Dreiecke}
		
	\begin{enumerate}
		\item Gegeben sei ein rechtwinkliges Dreieck $ABC$ mit den Katheten $a: = \overline{AB}$ und $b := \overline{AC}$. 
			\begin{enumerate}
				\item Welche Körper entstehen, wenn wir das Dreieck $ABC$ um $a$ bzw. $b$ rotieren lassen? 
				\item Geben Sie eine Vermutung ab, wie sich die beiden Volumina zueinander verhalten.
				\item Stellen Sie nun durch allgemeine Rechnung einen Zusammenhang zwischen den Rotationsvolumina $V_{AB}$ (Rotationsachse $AB$) und $V_{AC}$ (Rotationsachse $AC$) dar. \\ \emph{(Hinweis: Nutzen sie ein Gleichungssystem; Gesucht ist eine Proportionalität)}
				\item Von nun an gilt: $V=30 \pi$
					\begin{enumerate}
						\item Wie viele Wertepaare $(a,b)$ mit $a,b \in \mathbb{R}$ gibt es, so dass gilt $V=30 \pi$.
						\item Stellen Sie $b(a)$ (Länge b in Abhängigkeit von Länge a) dar.
						\item Zeichnen Sie $b(a)$ für $0 < a  \leq 10cm $ in ein Koordinatensystem.
					\end{enumerate} 	
				
			\end{enumerate}
				
	\end{enumerate} 
	
		

\end{document}