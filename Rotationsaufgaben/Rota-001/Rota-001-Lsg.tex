\documentclass{scrartcl}
\usepackage{amsmath}
\usepackage{a4wide}
\usepackage{float}
\usepackage[ngerman]{babel}
\usepackage{amssymb}
\usepackage{hyperref}
\usepackage[utf8]{inputenc}
\usepackage[T1]{fontenc}
\usepackage{graphicx}
\parindent0cm

\setcounter{tocdepth}{4}
\setcounter{secnumdepth}{4}
\usepackage[headsepline]{scrpage2}

\usepackage{scrpage2}
\pagestyle{scrheadings}
\clearscrheadfoot



\usepackage{tikz} 
\usetikzlibrary{shapes.misc}

\usepackage{pstricks}
\usepackage{pst-plot}
\usepackage{pstcol}



\ihead{Matthias Linhuber}
\chead{Arbeitsblatt Mathe}
\ohead{\today}

\usepackage{geometry}
\geometry{a4paper, top=20mm, left=20mm, right=20mm, bottom=10mm,
headsep=7mm, footskip=12mm}




\begin{document}

		
	\begin{enumerate}
		\item Gegeben sei ein Rechteck mit Kantenlängen $a$ und $b$. 
			\begin{enumerate}
				\item Kegel	
				\item Sei dem Schüler überlassen
				\item Die allgemeine Formel zur Berechnung des Volumens eines Kegels lautet:
				
					\begin{equation}
						V_{Kegel} =\frac{1}{3}\cdot r^2 \cdot \pi \cdot h 
					\end{equation}
					
					Wir wissen, dass die beiden Kegel die Radien $a$ und $b$ haben. 
					
					\begin{equation}
						\label{V_AB}
						V_{AB} =\frac{1}{3}\cdot b^2 \cdot \pi \cdot a 
					\end{equation}
					
					\begin{equation}
						\label{V_AC}
						V_{AC} =\frac{1}{3}\cdot a^2 \cdot \pi \cdot b 
					\end{equation}
					
					
					Nun gilt es das Gleichungssystem nach $a$ und $b$ zu lösen:\\
					
					Wir formen (\ref{V_AB}) nach $a$ um:
					
					\begin{equation}
						a = \frac{V_{AB} \cdot 3}{b^2 \cdot \pi} 
					\end{equation}
					
					und setzen $a$ in (\ref{V_AC}) ein:
					\begin{equation}
						V_{AC} =\frac{1}{3}\cdot \left (\frac{V_{AB} \cdot 3}{b^2 \cdot \pi} \right )^2 \cdot \pi \cdot b 
					\end{equation}
					
					Wir vereinfachen: 
					
					\begin{eqnarray*}
						V_{AC} &=& \frac{1}{3}\cdot \frac{V_{AB}^2 \cdot 9}{b^4 \cdot \pi^2}  \cdot \pi \cdot b \\
						V_{AC} &=& \frac{V_{AB}^2 \cdot 3}{b^3 \cdot \pi} \\
						V_{AC} &=& \frac{3}{b^3 \cdot \pi} \cdot  V_{AB}^2 \\
					\end{eqnarray*}
						
						
					
				\item Von nun an gilt: $V=30 \pi$
					\begin{enumerate}
						\item Unendlich Viele 
						\item Wir nutzen (\ref{V_AC}) um die Abhängigkeit darzustellen:
							
							\begin{eqnarray*}
								30 \pi &=& \frac{1}{3}\cdot a^2 \cdot \pi \cdot b \\
								90 &=& a^2 \cdot b \\
								b &=& \frac{90}{a^2}\\
								b(a) &=& \frac{90}{a^2}
							\end{eqnarray*}
							
								 
							
						\item Zeichnung:\\\\
							\begin{pspicture}[xAxisLabel=$x$,yAxisLabel=$y$](-0.5,0)(0.5,6.5)
								\begin{psgraph}[arrows=->,Dx=1,Dy=20](0,0)(-0.5,-5)(10,200){8cm}{6cm}
								    %\psplot[plotpoints=200,linecolor=blue]{0}{10}{40 0.2 x mul sub}
								    \psplot[plotpoints=200,linecolor=red]{0.69}{10}{90 x x mul div}
								\end{psgraph}
							\end{pspicture}
							
					\end{enumerate} 	
				
			\end{enumerate}
				
	\end{enumerate} 


		

\end{document}