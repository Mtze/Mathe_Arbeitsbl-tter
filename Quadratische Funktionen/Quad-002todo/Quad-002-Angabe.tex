\documentclass{scrartcl}
\usepackage{amsmath}
\usepackage{a4wide}
\usepackage{float}
\usepackage[ngerman]{babel}
\usepackage{amssymb}
\usepackage{hyperref}
\usepackage[utf8]{inputenc}
\usepackage[T1]{fontenc}
\usepackage{graphicx}
\parindent0cm

\setcounter{tocdepth}{4}
\setcounter{secnumdepth}{4}
\usepackage[headsepline]{scrpage2}

\usepackage{scrpage2}
\pagestyle{scrheadings}
\clearscrheadfoot

\ihead{Matthias Linhuber}
\chead{Arbeitsblatt Mathe}
\ohead{\today}

\usepackage{geometry}
\geometry{a4paper, top=20mm, left=20mm, right=20mm, bottom=10mm,
headsep=7mm, footskip=12mm}




\begin{document}

	\begin{enumerate}
		\item Gegeben seinen die Parabel $p(x) = a*(x-b)+c$, sowie die Punkte $A (-3,2)$, $B(5,-4)$, $C(x,p(x))$, $S(4,4)$ und  $Q(6,4)$.
			\begin{enumerate}
				\item Zeichnen Sie die Punkte $A$ und $B$ in ein passendes Koordinatensystem. \\ \emph{Hinweis: Achten Sie auf alle Punkte und deren Lage!}
				\item Berechnen Sie den Abstand zwischen den Punkten $A$ und $B$. \emph{Hinweis: Formelsammlung}
				\item Geben Sie die Geradengleichung $g(x)$ der Geraden durch $A$ und $B$ an.
				\item Geben Sie den Term der Parabel $p(x)$ an, für die gilt: $a=-1$ und $S,Q \in p$. \\ \emph{Hinweis: Überlegen Sie, wo der Scheitelpunkt der Parabel liegen muss. Alternativ: Gleichungssystem. \\ Ergebniss: $p(x) = -x^2 +10x-20$}
				\item Zeichnen sie die Parabel $p$ in das Koordinatensystem.
				\item Der Punkt $C$ bewegt sich nun auf der Parabel $p$. Geben Sie in Abhängigkeit von x den Abstand $\overline{AC}$ an.
				\item Berechnen Sie die Nullstellen der Parabel $p$.
				\item Geben Sie die Geradengleichung der Gerade $t$ an, die senkrecht auf $g$ steht und durch den Punkt $B$ verläuft. \\ \emph{Hinweis: Überlegen Sie wie sich die Steigungen von $t$ und $g$ verhalten müssen.}
				\item Zeichnen Sie die Gerade $t$ in das Koordinatensystem
				\item Berechnen Sie die Schnittpunkte $P_{1/2}$ von $p$ und $t$
				\item Berechnen Sie den Flächeninhalt des Dreiecks $ABP$
				\item Stellen Sie sich das Dreieck $ABC$ vor und überlegen Sie wie Sie den Flächeninhalt allgemein Berechnen könnten. (Nicht ausführen, nur nachdenken!)
			\end{enumerate}
		
	\end{enumerate}
			

\end{document}