\documentclass{scrartcl}
\usepackage{amsmath}
\usepackage{a4wide}
\usepackage{float}
\usepackage[ngerman]{babel}
\usepackage{amssymb}
\usepackage{hyperref}
\usepackage[utf8]{inputenc}
\usepackage[T1]{fontenc}
\usepackage{graphicx}
\parindent0cm

\setcounter{tocdepth}{4}
\setcounter{secnumdepth}{4}
\usepackage[headsepline]{scrpage2}

\usepackage{scrpage2}
\pagestyle{scrheadings}
\clearscrheadfoot

\ihead{Matthias Linhuber}
\chead{Arbeitsblatt Mathe}
\ohead{\today}

\usepackage{geometry}
\geometry{a4paper, top=20mm, left=20mm, right=20mm, bottom=10mm,
headsep=7mm, footskip=12mm}




\begin{document}

		
	\begin{enumerate}
		\item Gegeben sind die Punkte $A:(-1|-3)$,\; $B:(8|-3)$ und $C:(4|7)$, sowie die Geraden $g: y = -3$ und $h: x = 4$
		\begin{enumerate}
			\item Zeichnen Sie das Dreieck ABC sowie die beiden Geraden $g$ und $h$ in ein kartesisches Koordinatensystem der Größe:\quad $ -2\leq x \leq 14 $ \quad $-4 \leq y \leq 8$ 
			\item Berechnen Sie den Flächeninhalt des Dreiecks $ABC$.
			\item Der Punkt $B$ wandert nun auf $g$ um $x$ LE\footnote{LE = Längeneinheiten} in positiver $x$-Richtung, $C$ dagegen um $0,5x$ LE in negativer $y$-Richtung. Die neuen Punkte heißen $B'$ und $C'$. Geben Sie die Koordinaten von $B‘$ und $C‘$ in Abhängigkeit von $x$ an.
			\item Zeichnen Sie für $x = 4$ das Dreieck $AB'C'$ in das Koordinatensystem ein.
			\item (Kniffligere Aufgabe) Berechnen Sie den Flächeninhalt $A(x)$ der Dreiecke $AB'C'$ in Abhängigkeit von $x$.
			 \\
		\end{enumerate}
			
			
		
	\end{enumerate} 
	
	

\end{document}