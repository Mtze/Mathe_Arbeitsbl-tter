\documentclass{scrartcl}
\usepackage{amsmath}
\usepackage{a4wide}
\usepackage{float}
\usepackage[ngerman]{babel}
\usepackage{amssymb}
\usepackage{hyperref}
\usepackage[utf8]{inputenc}
\usepackage[T1]{fontenc}
\usepackage{graphicx}
\parindent0cm

\setcounter{tocdepth}{4}
\setcounter{secnumdepth}{4}
\usepackage[headsepline]{scrpage2}

\usepackage{scrpage2}
\pagestyle{scrheadings}
\clearscrheadfoot

\ihead{Matthias Linhuber}
\chead{Arbeitsblatt Mathe}
\ohead{\today}

\usepackage{geometry}
\geometry{a4paper, top=20mm, left=20mm, right=20mm, bottom=10mm,
headsep=7mm, footskip=12mm}




\usepackage{tikz} 
\usetikzlibrary{shapes.misc}

\usepackage{pstricks}
\usepackage{pst-plot}
\usepackage{pstcol}




\begin{document}

		
	\begin{enumerate}
		\item Gegeben sind zwei Geraden: \quad $g(x): y = - \frac{2}{3}  x + 4$ \quad $f(x): y = x-2$
			\begin{enumerate}
				\item Prüfen durch einsetzen: \\
					$g(9)=- \frac{2}{3} \cdot 9 +4 = -2 $ \quad Der Punkt $A$ liegt auf $G_g$. ($A \in G_g$)
				\item 
					\begin{eqnarray*}
						g(x) &=& f(x)\\
						- \frac{2}{3}  x + 4 &=& x-2 \quad \quad | -x; -4\\
						- \frac{5}{3} x &=& -6 \quad\quad | \cdot (- \frac{3}{5}) \\
						x &=& \frac{18}{5}
					\end{eqnarray*}
					Einetzen in $f(x)$:
					\begin{eqnarray*}
						f \left( \frac{18}{5} \right ) = \frac{18}{5}-2 = \frac{8}{5}
					\end{eqnarray*}
					
					Schnittpunkt von $G_g$ und $G_f$: $S = \left (\frac{18}{5},\frac{8}{5} \right)$
					
					
				\item  Koordinatensystem siehe Geogebra
				
					%\begin{pspicture}[](-4,-6)(-4,8)

					%	\psaxes[labels={x,y}]{->}(0,0)(-4,-4)(10,8)
					%	\psline[linewidth=2pt]{-}(0,0)(2,2)(4,0)
					%\end{pspicture}
				\item Aus der Formelsammlung: Punktsteigungsformel: $h(x) = m \cdot (x - x_0 )+ y_0$ \\
					Auf konkretes Beispiel angewandt: $h(x) = m \cdot (x-9)-2$
				\item 
					\begin{eqnarray*}
						h(x)&=&f(x)\\
						m \cdot (x-9)-2 &=& x-2\\
						m \cdot (x-9) &=& x\\
						mx -9m &=& x \\
						-9m &=& x-mx \\
						-9m &=& x(1-m) \\
						\frac{-9m}{1-m} &=& x
					\end{eqnarray*} \\
					Einsetzen in $f(x)$
					\begin{eqnarray*}
						f\left(\frac{-9m}{1-m}\right) = \frac{-9m}{1-m} -2 
					\end{eqnarray*}
					Schnittpunkt von $G_h$ und $G_f$: $S = \left (\frac{-9m}{1-m},\frac{-9m}{1-m} -2  \right)$
			\end{enumerate}
			
			
	\end{enumerate} 
	
	

\end{document}