\documentclass{scrartcl}
\usepackage{amsmath}
\usepackage{a4wide}
\usepackage{float}
\usepackage[ngerman]{babel}
\usepackage{amssymb}
\usepackage{hyperref}
\usepackage[utf8]{inputenc}
\usepackage[T1]{fontenc}
\usepackage{graphicx}
\parindent0cm

\setcounter{tocdepth}{4}
\setcounter{secnumdepth}{4}
\usepackage[headsepline]{scrpage2}

\usepackage{scrpage2}
\pagestyle{scrheadings}
\clearscrheadfoot

\ihead{Matthias Linhuber}
\chead{Arbeitsblatt Mathe}
\ohead{\today}

\usepackage{geometry}
\geometry{a4paper, top=20mm, left=20mm, right=20mm, bottom=10mm,
headsep=7mm, footskip=12mm}




\begin{document}

		
	\begin{enumerate}
		\item Gegeben sind zwei Geraden: \quad $g(x): y = - \frac{2}{3}  x + 4$ \quad $f(x): y = x-2$
			\begin{enumerate}
				\item Überprüfen Sie, ob der Punkt $A: (9|-2)$ auf einer der Geraden liegt.
				\item Berechnen Sie den Schnittpunkt der beiden Geraden.
				\item Legen Sie ein kartesisches Koordinatensystem der Größe:\quad $ -4 \leq x \leq 10 $ \quad $-4 \leq y \leq 8$  an und zeichnen Sie die beiden Geraden, sowie den Punkt A ein.
				\item Geben Sie die Geradengleichung der Geradenschar $h(x)$ an, die durch den Punkt A läuft. (Hinweis: Die Steigung ist in dieser Gleichung variabel)
				\item (Kniffligere Aufgabe) Geben Sie, in Abhängigkeit von $m$, die Koordinaten der Schnittpunkte von $h(x)$ und $f(x)$ an und begründen Sie welche Werte für $m$ sinnvoll sind. \\
			\end{enumerate}
			
			
	\end{enumerate} 
	
	

\end{document}