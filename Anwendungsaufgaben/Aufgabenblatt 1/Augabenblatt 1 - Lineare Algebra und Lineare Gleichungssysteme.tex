\documentclass{scrartcl}
\usepackage{amsmath}
\usepackage{a4wide}
\usepackage{float}
\usepackage[ngerman]{babel}
\usepackage{amssymb}
\usepackage{hyperref}
\usepackage[utf8]{inputenc}
\usepackage[T1]{fontenc}
\usepackage{graphicx}
\parindent0cm

\setcounter{tocdepth}{4}
\setcounter{secnumdepth}{4}
\usepackage[headsepline]{scrpage2}

\usepackage{scrpage2}
\pagestyle{scrheadings}
\clearscrheadfoot

\ihead{Matthias Linhuber}
\chead{Arbeitsblatt Mathe}
\ohead{\today}

\usepackage{geometry}
\geometry{a4paper, top=20mm, left=20mm, right=20mm, bottom=10mm,
headsep=7mm, footskip=12mm}




\begin{document}

	\section{Lineare Algebra, Lineare Gleichungssysteme und Flächengeometrie}
		
	\begin{enumerate}
		\item Gegeben sind zwei Geraden: \quad $g(x): y = - \frac{2}{3}  x + 4$ \quad $f(x): y = x-2$
			\begin{enumerate}
				\item Überprüfen Sie, ob der Punkt $A: (9|-2)$ auf einer der Geraden liegt.
				\item Berechnen Sie den Schnittpunkt der beiden Geraden.
				\item Legen Sie ein kartesisches Koordinatensystem der Größe:\quad $ -4 \leq x \leq 10 $ \quad $-4 \leq y \leq 8$  an und zeichnen Sie die beiden Geraden, sowie den Punkt A ein.
				\item Geben Sie die Geradengleichung der Geradenschar $h(x)$ an, die durch den Punkt A läuft. (Hinweis: Die Steigung ist in dieser Gleichung variabel)
				\item (Kniffligere Aufgabe) Geben Sie, in Abhängigkeit von $m$, die Koordinaten der Schnittpunkte von $h(x)$ und $f(x)$ an und begründen Sie welche Werte für $m$ sinnvoll sind. \\
			\end{enumerate}
			
			
		\item Gegeben sind die Punkte $A:(-1|-3)$,\; $B:(8|-3)$ und $C:(4|7)$, sowie die Geraden $g: y = -3$ und $h: x = 4$
			\begin{enumerate}
				\item Zeichnen Sie das Dreieck ABC sowie die beiden Geraden g und h in ein kartesisches Koordinatensystem der Größe:\quad $ -2\leq x \leq 14 $ \quad $-4 \leq y \leq 8$ 
				\item Berechnen Sie den Flächeninhalt des Dreiecks ABC.
				\item Der Punkt B wandert nun auf g um x LE\footnote{LE = Längeneinheiten} in positiver x-Richtung, C dagegen um 0,5x LE in negativer y-Richtung. Die neuen Punkte heißen B' und C'. Geben Sie die Koordinaten von B‘ und C‘ in Abhängigkeit von x an.
				\item Zeichnen Sie für x = 4 das Dreieck AB'C' in das Koordinatensystem ein.
				\item (Kniffligere Aufgabe) Berechnen Sie den Flächeninhalt $A(x)$ der Dreiecke AB'C' in Abhängigkeit von x.
				 \\
			\end{enumerate}
			
			
		\item Lösen Sie folgende Gleichungssysteme. (Ist kein Lösungsverfahren angegeben, wählen Sie geschickt, auf welche Art Sie die Aufgabe lösen möchten!):
			\begin{enumerate}
				\item $ \frac{4}{6}x + 6y = 0 \quad \wedge \quad \frac{3}{2}x - \frac{1}{2}y =3$
				\item $ 2x + 6y = 0 \quad \wedge \quad 2x - 27y =15$
				\item $ \frac{3}{5}x - \frac{5}{6} y + 15= 0 \quad \wedge \quad \frac{7}{5}x = \frac{1}{3}y -10$ \quad (Determinantenverfahren)
				\item $7x = \frac{1}{4}y +3 \quad \wedge \quad 3x -4 = y$
				 \\
			\end{enumerate}
		\item $A (7|4)$ und $C(2|9)$ sind Eckpunkte einer Raute $ABCD$ mit $A = 40$ FE\footnote{FE = Flächeneinheiten}. Berechnen Sie Länge der Diagonalen [BD].
		\item Wenden Sie die Binomischen Formeln sinnvoll an:
				\begin{enumerate} 
					\item $(2+b)^2$
					\item $4x^2+12x+9$
					\item $(4x -9y)^2$
					\item $(16x^2 -72xy +81y^2)$
				\end{enumerate}
		\item Vervollständigen Sie folgende Tabelle: \\ \\
				\begin{tabular}[c]{|l|c|c|c|c|c|}
				\hline
				Kathete a    & 7 &   & 12 & 12 & 15 \\ \hline
				Kathete b    & 5 & 2 &    & 8  & 15 \\ \hline
				Hypotenuse c &   & 6 & 13 &    &    \\ \hline
				\end{tabular}
	\end{enumerate} 
	
	

\end{document}