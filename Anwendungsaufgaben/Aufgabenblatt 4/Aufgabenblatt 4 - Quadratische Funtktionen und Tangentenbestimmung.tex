\documentclass{scrartcl}
\usepackage{amsmath}
\usepackage{a4wide}
\usepackage{float}
\usepackage[ngerman]{babel}
\usepackage{amssymb}
\usepackage{hyperref}
\usepackage[utf8]{inputenc}
\usepackage[T1]{fontenc}
\usepackage{graphicx}
\parindent0cm

\setcounter{tocdepth}{4}
\setcounter{secnumdepth}{4}
\usepackage[headsepline]{scrpage2}

\usepackage{scrpage2}
\pagestyle{scrheadings}
\clearscrheadfoot



\usepackage{tikz} 
\usetikzlibrary{shapes.misc}

\usepackage{pstricks}
\usepackage{pst-plot}
\usepackage{pstcol}



\ihead{Matthias Linhuber}
\chead{Arbeitsblatt Mathe}
\ohead{\today}

\usepackage{geometry}
\geometry{a4paper, top=20mm, left=20mm, right=20mm, bottom=10mm,
headsep=7mm, footskip=12mm}




\begin{document}

	\section{Vermischte Übung 9. Klasse}
		
	\begin{enumerate}
		\item Gegeben seien die Punkte $A(0,0)$, $B(-2,-4)$ und $C(x,f(x))$. \\Weiterhin ist die Funktion $f(x):=x^2 +x +2$ gegeben.
			\begin{enumerate}
				\item Geben Sie die Definitions- und Wertemenge von $f$ an.
				\item Zeichnen Sie $G_f$ in ein kartesisches Koordinatensystem.
				\item Berechnen Sie die längen $\overline{AB}$ und $\overline{AC}$.
				\item Beschreiben Sie in Prosa, wie sich der Punkt $C$ bewegt.
				\item Berechnen Sie die Fläche des Dreiecks $ABC$.
				\item Der Flächeninhalt des Dreiecks $ABC$ beträgt nun $10FE$. Geben sie mögliche Koordinaten von $C$ an.
				\item Legen Sie an $G_f$ die Tangenten durch $A$ an.
			\end{enumerate}
				
	\end{enumerate} 
	
	
	\section{Lösung}
		\begin{enumerate}
			\item 
				\begin{enumerate}
					\item $\mathbb{D} = \mathbb{R}$, \quad $\mathbb{W} = \lbrace y | y \geq 2 \rbrace$
					\item Zeichnung: 
						\begin{pspicture}[](-6,-6)(-4,18)

							\psaxes[labels={x,y}]{->}(0,0)(-6,0)(6,18)

						\end{pspicture}
				\end{enumerate}
		\end{enumerate}
	
	

\end{document}