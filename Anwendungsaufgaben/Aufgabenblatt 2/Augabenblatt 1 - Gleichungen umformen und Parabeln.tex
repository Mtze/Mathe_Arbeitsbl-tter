\documentclass{scrartcl}
\usepackage{amsmath}
\usepackage{a4wide}
\usepackage{float}
\usepackage[ngerman]{babel}
\usepackage{amssymb}
\usepackage{hyperref}
\usepackage[utf8]{inputenc}
\usepackage[T1]{fontenc}
\usepackage{graphicx}
\parindent0cm

\setcounter{tocdepth}{4}
\setcounter{secnumdepth}{4}
\usepackage[headsepline]{scrpage2}

\usepackage{scrpage2}
\pagestyle{scrheadings}
\clearscrheadfoot

\ihead{Matthias Linhuber}
\chead{Arbeitsblatt Mathe}
\ohead{\today}

\usepackage{geometry}
\geometry{a4paper, top=20mm, left=20mm, right=20mm, bottom=10mm,
headsep=7mm, footskip=12mm}




\begin{document}

	\section{Gleichungen umformen}
			\begin{enumerate}
				\item Formen Sie folgende Gleichungen nach der angegebenen Variable um:
				\begin{enumerate}
					\item $3*x + (x-3)^2 = -x-3^2*x$ \quad Umformung nach $x$ 
					\item $(2*y-y)^2+3x=(a+b)*y^2$ \quad Umformung nach $y$ 
					\item (Zusatzaufgabe) $\frac{x}{y^2}*a = b+x*c$ \quad Umformung nach $y$\\
				\end{enumerate}
		\end{enumerate} 
	
	\section{Parabelfunktionen}		
		\begin{enumerate}
			\item Gegeben sind die Parabeln $p(x) = a*x^2 + t$ \quad \quad $a \in \mathbb{N}; t \in \mathbb{Z}$
				\begin{enumerate}
					\item Geben Sie die Definitions- und Wertemenge von $p$, sowie die vollständige Abbildungsvorschrift an
					\item Zeichnen Sie den Graphen der Parabel $p$ für $a=2$ und $t = -1$ in ein kartesisches Koordinatensystem der Größe:\quad $ -5 \leq x \leq 5 $ \quad $-4 \leq x \leq 5$ 
					\item Berechnen Sie die Koordinaten der Nullstellen der Parabel $p$. ($a$ und $t$ wie oben)
					\item Berechnen Sie nun die Koordinaten der Nullstelle in Abhängigkeit von a und t \\
				\end{enumerate}	
		\end{enumerate} 

\end{document}