\documentclass{scrartcl}
\usepackage{amsmath}
\usepackage{a4wide}
\usepackage{float}
\usepackage[ngerman]{babel}
\usepackage{amssymb}
\usepackage{hyperref}
\usepackage[utf8]{inputenc}
\usepackage[T1]{fontenc}
\usepackage{graphicx}
\parindent0cm

\setcounter{tocdepth}{4}
\setcounter{secnumdepth}{4}
\usepackage[headsepline]{scrpage2}

\usepackage{scrpage2}
\pagestyle{scrheadings}
\clearscrheadfoot

\ihead{Matthias Linhuber}
\chead{Arbeitsblatt Mathe}
\ohead{\today}

\usepackage{geometry}
\geometry{a4paper, top=20mm, left=20mm, right=20mm, bottom=10mm,
headsep=7mm, footskip=12mm}




\begin{document}

		
	\begin{enumerate}
		
		
		\item Gegeben Seien Punkt $B(0,4)$ und Parabel $p(x):=(x-5)^2+s$.
		\begin{enumerate}
			\item Tangente durch den Punkt $B$ $\Rightarrow B \in G_q$ \label{Tangentengleichung_bestimmen} \\
				\\ Punkt-Steigungs-Gleichung:
				\begin{eqnarray}
					q_{1/2}(x) &=& m_{1/2} \cdot (x-x_0)+y_0 \\
					q_{1/2}(x) &=& m_{1/2} \cdot (x-0)+4 \\
					q_{1/2}(x) &=& m_{1/2} \cdot x+4 \\
				\end{eqnarray}
				$q_{1/2}$ mit $p$ Gleichsetzen:
				\begin{eqnarray}
					q_{1/2}(x) &=& p(x)\\
					m_{1/2} \cdot x+4 &=& (x-5)^2+s \\
					m_{1/2} \cdot x+4 &=& (x^2-10x +25)+s\\
					0 &=& x^2-10x-m_{1/2}\cdot x+21+s\\
					0 &=& x^2+(-10-m_{1/2})x+21+s \label{Ende_Normalform_x}
				\end{eqnarray}
				Anwendung der Mitternachtsformel für $x_{1/2}$: $a=1$\ ;\ $b=(-10-m_{1/2})$\ ;\ $c= 21+s$
				\begin{eqnarray}
					D_x &=& b^2-4 \cdot a \cdot c \\
					D_x &=& (-10-m_{1/2})^2 -4 \cdot 1 \cdot (21+s)\\
					D_x &=& (m_{1/2}^2+20m_{1/2}+100) -84-4s\\
					D_x &=& m_{1/2}^2+20m_{1/2}+16-4s \label{Ende_D_x}
				\end{eqnarray}
				
				
				
				
				Da eine Tangente gesucht ist dürfen nur Berührpunkte vorhanden sein $\Rightarrow D_x = 0$:
				\begin{eqnarray}
					D_x &=& 0 \\
					0 &=& m_{1/2}^2+20m_{1/2}+16-4s   
				\end{eqnarray}
				Anwendung der Mitternachtsformel für $m_{1/2}$ $a=1$\ ;\ $b=20$\ ;\ $c= 16-4s$
				\begin{eqnarray}
					D_m &=& b^2 - 4 \cdot a \cdot c \\
					D_m &=& 20^2 - 4 \cdot 1 \cdot (16-4s) \\
					D_m &=& 336+16s \label{Ende_D_m}
				\end{eqnarray}
				Einsetzen in Mitternachtsformel:
				\begin{eqnarray}
					m_{1/2} &=& \frac{-b \pm \sqrt{D}}{2 \cdot a} \\
					m_{1/2} &=& \frac{-20 \pm \sqrt{336+16s}}{2} \\
					m_{1/2} &=& \frac{-20 \pm \sqrt{16\cdot (21+s)}}{2}\\
					m_{1/2} &=& \frac{-20 \pm 4 \cdot \sqrt{21+s}}{2}\\
					m_{1/2} &=& -10 \pm 2 \cdot \sqrt{21+s} 
				\end{eqnarray}
				
				Die beiden Steigungen werden nun in $q(x)$ eingesetzt:
				
				\begin{eqnarray}
					q_{1/2}(x) &=& m_{1/2} \cdot x+4 \\
					q_{1/2}(x) &=& (-10 \pm 2 \cdot \sqrt{21+s})x +4 \\
				\end{eqnarray}
				
				Die beiden Funktionen lauten:
				
				\begin{equation}
					q_{1}(x) = (-10 + 2 \cdot \sqrt{21+s}) \cdot x + 4  = 2 \cdot (\sqrt{21+s} - 5) \cdot x + 4 
				\end{equation}
				und
				\begin{equation}
					q_{2}(x) = (-10 - 2 \cdot \sqrt{21+s})\cdot x + 4  = -2 \cdot (\sqrt{21+s} + 5) \cdot x + 4 
				\end{equation}
				
				
			\item Es gilt $B$ in $p(x) $ ein zu setzen:
				\begin{eqnarray}
					p(0)&=&4 \\
					4 &=& (0-5)^2+s \\
					4 &=& 25 + s\\
					-21 &=& s
				\end{eqnarray}
			\item Im Folgenden Sei $s:=-2$.
			\item Siehe GGB

			\item Naiver Ansatz: $p(x) = q_1(x)$ und $p(x) = q_2(x)$. \\
			\item 
				Der erste Teil der Rechnung ist bereits in Aufgabe \ref{Tangentengleichung_bestimmen} beschrieben. Die Rechung wird nun nach (\ref{Ende_D_m}) weitergeführt: \\
				
				Zunächst das gegebene $s$ einsetzen:
				\begin{eqnarray}
					m_{1/2} &=& -10 \pm 2 \cdot \sqrt{21+s} \\
					m_{1/2} &=& -10 \pm 2 \cdot \sqrt{21-2} \\
					m_{1/2} &=& -10 \pm 2 \cdot \sqrt{19} \\
					m_1 &=& -10 + 2 \cdot \sqrt{19} \label{m_1} \\
					m_2 &=& -10 - 2 \cdot \sqrt{19}
				\end{eqnarray}
				
				Die nun explizite Belegung von $m$ und $s$ kann in (\ref{Ende_Normalform_x}) eingesetzt werden. \\
				
				Zunächst $m_1$ (Also Berührpunkt $g$ und $q_1$)
				\begin{eqnarray}
					0 &=& x^2+(-10-m_{1})\cdot x+21+s \\
					0 &=& x^2+(-10- (-10 + 2 \cdot \sqrt{19}))\cdot x+21-2 \\
					0 &=& x^2+(- 2 \cdot \sqrt{19})\cdot x+19
				\end{eqnarray}
				Trivialerweise gilt $D = 0$, da wir $m$ so Bestimmt haben.
				
				\begin{eqnarray}
					x&=& \frac{-b}{2 \cdot a} \\
					x&=& \frac{-(-2 \cdot \sqrt{19})}{2} \\
					x&=& \sqrt{19}
				\end{eqnarray}
				
				Analoge Berechnung für $m_2$ (Berührpunkt $g$ und $q_2$):
				\begin{eqnarray}
					0 &=& x^2+(-10-m_{2})\cdot x+21+s \\
					0 &=& x^2+(-10- (-10 - 2 \cdot \sqrt{19}))\cdot x+21-2 \\
					0 &=& x^2+(2 \cdot \sqrt{19})\cdot x+19
				\end{eqnarray}
				
				\begin{eqnarray}
					x&=& \frac{-b}{2 \cdot a} \\
					x&=& \frac{-(2 \cdot \sqrt{19})}{2} \\
					x&=& -\sqrt{19}
				\end{eqnarray}
				
				
				
				
				
				
			\item Zwei geraden stehen Senkrecht aufeinander wenn gilt: $m_a \cdot m_b = -1$.\\

				Senkrechte auf $q_1$: 
				\begin{equation}
					h(x) = m \cdot x + t \label{h(x)_alg}
				\end{equation}
				Laut (\ref{m_1}) gilt $m_1 = -10 + 2 \cdot \sqrt{19}$. Wir definieren nun  $m_a := m_1$
				\begin{eqnarray}
					m_a \cdot m_b &=& -1 \\
					m_b &=& - \frac{1}{m_a} \\
					m_b &=& - \frac{1}{-10 + 2 \cdot \sqrt{19}} \label{m_b}
				\end{eqnarray}
				
				Nun setzen wir (\ref{m_b}) in (\ref{h(x)_alg}) ein:
				
				\begin{eqnarray}
					h(x) &=& m_b \cdot x + t \\
					h(x) &=& \left (- \frac{1}{-10 + 2 \cdot \sqrt{19}} \right ) \cdot x + t 
				\end{eqnarray}
				
				Rechenweg analog für zweite Senkrechte.
				
			\item Der Punkt $N(0,t)$ Liegt auf $G_p$ wenn gilt: \label{Schnittpunkt_h_p}
				\begin{equation}
					p(0) = t
				\end{equation}
				Wir berechnen also:
				\begin{eqnarray}
					t &=& (0-5)^2-2 \\
					t &=& 23
				\end{eqnarray}
			
		\end{enumerate} % Ende Aufgabe 2
		
		
		
	\end{enumerate} 
	
	
	
	
	
	
	
	
	
	
	
	
	
	
	
	
	
	
	
	
	
	
	
	
	
	
	
	
	
	
	
	
	
	
	
	
	
	
	
	
	
	

\end{document}