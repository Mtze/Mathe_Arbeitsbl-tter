\documentclass{scrartcl}
\usepackage{amsmath}
\usepackage{a4wide}
\usepackage{float}
\usepackage[ngerman]{babel}
\usepackage{amssymb}
\usepackage{hyperref}
\usepackage[utf8]{inputenc}
\usepackage[T1]{fontenc}
\usepackage{graphicx}
\parindent0cm

\setcounter{tocdepth}{4}
\setcounter{secnumdepth}{4}
\usepackage[headsepline]{scrpage2}

\usepackage{scrpage2}
\pagestyle{scrheadings}
\clearscrheadfoot

\ihead{Matthias Linhuber}
\chead{Arbeitsblatt Mathe}
\ohead{\today}

\usepackage{geometry}
\geometry{a4paper, top=20mm, left=20mm, right=20mm, bottom=10mm,
headsep=7mm, footskip=12mm}




\begin{document}

	\section{Quadratische Funktionen und Tangentenbestimmung}
		
	\begin{enumerate}
		\item Gegeben seien $f(x) := (x-3)^2+3$, $p(x) := (x+3)^2+3$ und $h(x) := -x^2+8$
		\begin{enumerate}
			\item Skizzieren Sie den Graphen der Parabeln (Keine genaue Zeichnung notwending).
			\item Berechnen Sie die Schnittpunkte der Parabeln $p$ und $h$.
			\item Gegeben ist nun weiter der Punkt $A(0,10)$. Bestimmen Sie, durch gut dokumentierte Rechnung, EINE der Tangenten durch den Punkt $A$ an den Parabeln\footnote{Die Schreibweise $G_f$ bedeutet "Graph von f"} $G_f$, $G_p$ oder $G_h$. \\
		\end{enumerate} %Ende Aufgabe 1 
			
	\end{enumerate} 
	
	

\end{document}